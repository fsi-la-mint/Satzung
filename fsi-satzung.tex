\documentclass[12pt]{article}
\usepackage[a4paper, left=2.5cm, right=2.5cm, top=3.5cm, bottom=3.5cm]{geometry}
\usepackage[utf8]{inputenc}
\usepackage[ngerman]{babel}
\usepackage{amsmath}
\usepackage{amssymb}
\usepackage{enumitem}
\usepackage{graphicx}

\renewcommand{\familydefault}{\sfdefault} % serifenlose Schrift

\usepackage{xcolor}

%\newcommand{\red}[1]{\textcolor{red}{#1}}
%\newcommand{\gray}[1]{\textcolor{gray}{#1}}
%\newcommand{\green}[1]{\textcolor{green}{#1}}
%\newcommand{\orange}[1]{\textcolor{orange}{#1}}

\usepackage[hidelinks]{hyperref}

% Definiere hier das Datum des Inkrafttretens der Satzung:
\newcommand{\satzungsdatum}{05.03.2025}


\title{Satzung der FSI Lehramt MINT}

\date{\satzungsdatum}

\begin{document}

\maketitle

\section*{§1. Name}

\begin{enumerate}
    \item Der Name der Fachschaft wurde auf FSI Lehramt MINT festgelegt.
    \item Es wurde sich auf die Abkürzung FSI LA-MINT geeinigt.
\end{enumerate}

\section*{§2. Logo}

	Es wird folgendes Logo (und dessen Abwandlungen) verwendet:
	\begin{figure}[h]
		\includegraphics[width=0.5\textwidth]{img/logo_MINT.png}
	\end{figure}

\section*{§3. Zweck der FSI}

\begin{enumerate}
	\item
		\label{3.1}
		Die FSI soll eine Anlaufstelle für die Studierenden der Fächer Lehramt Biologie, Lehramt Chemie, Lehramt Informatik, Lehramt Mathematik und Lehramt Physik
		darstellen und diese während ihrer Studienzeit
		unterstützen. Dazu gehört vor allem die Einzelberatung (z.B.
		Eignung und Schwierigkeit einzelner Module) und die allgemeine
		Betreuung von Studierenden (also auch Spaß).
	\item
		Die FSI soll die Studierenden der in \hyperref[3.1]{§3.1} genannten Fächer
		repräsentieren, auf akute und langfristige Mängel im
		Studienverlauf aufmerksam machen und somit die betreffenden Studienfächer aus der Sicht von Studierenden vertreten.
	\item
		Die FSI soll für die Studierenden der in \hyperref[3.1]{§3.1} genannten Studienfächer
		ein Angebot an Aktivitäten, Treffen und Gemeinschaft
		ermöglichen.
	\item
		Es können nur diejenigen der in \hyperref[3.1]{§3.1} genannten Fächer durch die FSI vertreten werden, von denen auch Studierende in der FSI Mitglied sind.
\end{enumerate}

\section*{§4. Rechtsgrundlage}

\begin{enumerate}
	\item
		Die FSI ist kein eingetragener Verein und ist daher weder eine
		juristische Person noch rechtlich verankert.
	\item
		Diese Satzung wurde als Konsens erstellt und hält die Ziele,
		Absichten und Abläufe in der FSI fest. Sie ist jedoch nicht
		rechtlich bindend.
    	\item
		Die FSI ist offen für alle Interessierten, unabhängig von deren
		Staatsangehörigkeit, Geschlecht, Herkunft, Religion,
		Weltanschauung, Parteizugehörigkeit und gesellschaftlichen
		Stellung.
\end{enumerate}

\section*{§5. Mitgliedschaft}

\begin{enumerate}
	\item
		Es sind grundsätzlich alle Studierenden, ehemalige Studierende
		und Interessierte zu den Veranstaltungen und Aktionen
		eingeladen, falls es nicht explizit anders im Voraus gemeldet
		wurde.
    	\item
		Alle Studierenden der in \hyperref[3.1]{§3.1} genannten Studienfächer, welche innerhalb
		eines Jahres mindestens drei oder mehr Veranstaltungen
		teilnehmen, können die Mitgliedschaft erhalten. Alternativ kann
		die Mitgliedschaft auch durch einen einstimmigen
		Mitgliederbeschluss erhalten werden.
	\item
		Eine aktive Mitgliedschaft ist genau dann vorhanden, wenn das
		Mitglied unter den Sitzungen der letzten 6 Monate bei mindestens
		50\% anwesend oder vorab entschuldigt war.
	\item
		Mitgliedschaften können durch das Mitglied selbst niedergelegt
		werden.
	\item 	Mitgliedschaften können durch den Vorstand aufgehoben werden.
\end{enumerate}

\section*{§6. Organe der FSI}

\begin{enumerate}
	\item
		Der Vorstand
	\item
		Die FSI-Sitzung
	\item
		Die Mitgliederversammlung
\end{enumerate}

\section*{§7. Der Vorstand}

\begin{enumerate}
		\item
		Der Vorstand besteht aus mindestens einer/einem Vorsitzenden, einer/einem stellvertretenden Vorsitzenden. Zudem ist jeweils die Studienfachvertretung jeder Fachrichtung, die derzeit von der FSI vertreten wird, automatisch Mitglied im Vorstand.
    	\item
		Der Vorstand hat vor allem folgende Aufgaben:
    		\begin{enumerate}[label=(\roman*)]
			\item
				Vorbereiten und Einberufen sowie das Aufstellen von
				Tagesordnungspunkten der Mitgliederversammlungen und Sitzungen
        		\item
				Abschluss und Kündigung von Verträgen
        		\item
				Mitgliederverwaltung
    		\end{enumerate}
\end{enumerate}

\section*{§8. Bestellung des Vorstands}

\begin{enumerate}
	\item
		Die Vorsitzenden können nur aktive Mitglieder sein.
	\item
	Studienfachvertretung:
	\begin{enumerate}[label=(\roman*)]
		\item
		Jedes Studienfach, welches derzeit von der FSI vertreten wird,
		besitzt eine Studienfachvertreterin oder einen
		Studienfachvertreter, die/der das entsprechende Studienfach
		derzeit studiert.
		\item
		Die Studienfachvertretung wird per relativer Mehrheit aller derzeit das jeweilige Studienfach studierenden FSI-Mitglieder gewählt.
		\item
		Mehrfachbesetzungen aus Studienfachvertretung eines oder mehrerer Fächer und/oder (stellvertretendem) FSI-Vorsitz sind zulässig.
	\end{enumerate}
	\item
		Alle Mitglieder des Vorstandes werden während einer
		Mitgliederversammlung für maximal 1 Jahr gewählt. Jene bleiben
		auch nach einer regulären Amtszeit bis zur nächsten gültigen
		Wahl im Amt.
	\item
		Wiederwahlen sind zulässig.
\end{enumerate}

\section*{§9. Beschlussfassung des Vorstands}

\begin{enumerate}
    	\item
		Der Vorstand fasst seine Beschlüsse im Allgemeinen in
		Vorstandssitzungen, die schriftlich, fernmündlich oder in
		Textform (bspw. E-Mail) einberufen werden. Jedes
		Vorstandsmitglied ist einberufungsberechtigt.
    		\begin{enumerate}[label=(\roman*)]
			\item
				Eine Einberufungsfrist von einer Woche soll
				eingehalten werden. In dringenden Fällen ist
				eine Einberufung mit kürzerer Frist zulässig.
			\item
				Der Vorstand ist beschlussfähig, wenn
				mindestens zwei Vorstandsmitglieder anwesend
				sind oder schriftlich zustimmen.
			\item
				Bei der Beschlussfassung entscheidet die
				Mehrheit der abgegebenen gültigen Stimmen. Bei
				Stimmengleichheit gilt der Antrag als abgelehnt.
    		\end{enumerate}
	\item
		Vorstandsbeschlüsse können auch schriftlich, fernmündlich oder
		in Textform (bspw. E-Mail) gefasst werden, wenn die absolute
		Mehrheit der Vorstandsmitglieder ihre Zustimmung zu der zu
		beschließenden Regelung erklären.
\end{enumerate}


\section*{§10. Die FSI-Sitzung}

	Die FSI-Sitzung ist für das „Tagesgeschäft“ und die
	dazugehörigen Entscheidungen zuständig.


\section*{§11. Einberufung der FSI-Sitzung}

\begin{enumerate}
    	\item
		Die FSI-Sitzung soll regelmäßig stattfinden.
    	\item
		Die FSI-Sitzung wird vom Vorstand schriftlich, fernschriftlich
		oder in Textform unter Verwendung elektronischer
		Kommunikationsmittel einberufen. Er wird dazu angehalten, dies
		mindestens eine Woche im Voraus zu tun.
    	\item
		Die Tagesordnung setzt der Vorstand fest. Allerdings können
		aktive Mitglieder, ebenso wie externe Gäste und Interessierte,
		Beiträge/Ergänzungen zur Tagesordnung über den Vorstand
		einbringen. Diese müssen innerhalb der nächsten 2 Sitzungen
		berücksichtigt werden.
\end{enumerate}



\section*{§12. Beschlussfassung der FSI-Sitzung}
\label{12}
\begin{enumerate}

	\item
	Eine FSI-Sitzung wird von einem der Vorstandsmitglieder geleitet, typischerweise einer/einem Vorsitzenden. Ist kein Vorstandsmitglied anwesend,
	bestimmen die anwesenden FSI-Mitglieder die/den Leitenden.
	\item
	\begin{enumerate}[label=(\roman*)]
		\item
		Sofern nicht anders geregelt, hat jedes FSI-Mitglied eine Stimme.
		\item
		Zur Ausübung des Stimmrechts kann auch ein
		anderes Mitglied schriftlich oder in Textform
		(bspw. E-Mail) bevollmächtigt werden.
		\item
		Die Bevollmächtigung ist für jede
		FSI-Sitzung gesondert zu erteilen.
		\item
		Ein Mitglied darf nicht mehr als drei
		fremde Stimmen vertreten.
	\end{enumerate}
	\item
	Die/der Protokollführende wird von der Sitzungsleitung bestimmt.
	Diese/dieser muss kein aktives Mitglied sein.
	\item
	Die Art der Abstimmung wird durch die Sitzungsleitung bestimmt. Die Art der Abstimmung
	ist, falls nicht anders bestimmt, offen. Auf Antrag eines
	Mitgliedes ist geheim abzustimmen.
	\item
	Die FSI-Sitzung ist beschlussfähig, wenn 50\%
	aller stimmberechtigten Mitglieder anwesend oder
	ordnungsgemäß vertreten sind. Besteht keine Beschlussfähigkeit,
	so können keine Beschlüsse gefasst werden.
	\item
	Fassen von Beschlüssen:
	\begin{enumerate}[label=(\roman*)]
		\item
		Im Allgemeinen wird ein Beschluss mit einfacher
		Mehrheit der abgegebenen gültigen Stimmen
		gefasst.
		\item
		Enthaltungen gelten nicht als abgegebene Stimme.
		\item
		Beschlüsse über Satzungsänderungen benötigen
		eine Zweidrittelmehrheit der abgegebenen
		Stimmen der aktiven Mitglieder.
		\item
		Die Abwahl eines Vorsitzenden benötigt
		ebenfalls eine Zweidrittelmehrheit der
		abgegebenen Stimmen der aktiven Mitglieder.
		\item
			Die Abwahl einer Studienfachvertretung benötigt
			ebenfalls eine Zweidrittelmehrheit der
			abgegebenen Stimmen der Mitglieder des
			entsprechenden Studiengangs.
	\end{enumerate}
	\item
	Über die Beschlüsse ist ein Protokoll anzufertigen, welches für
	Mitglieder einsehbar abgelegt oder per Mail zugesandt wird.
	Es soll enthalten:
	\begin{enumerate}[label=(\roman*)]
		\item
		Ort und Zeit der Versammlung
		\item
		Anwesende Personen sowie Zahl der erschienenen
		Mitglieder
		\item
		Die Abstimmungsergebnisse und Beschlüsse
		\item
		Bei Satzungsänderungen soll der genaue Wortlaut
		der Änderungen angegeben werden
	\end{enumerate}
\end{enumerate}
\section*{§13. Satzungsänderungen}
\begin{enumerate}
	\item
	Soll eine Satzungsänderung während einer FSI-Sitzung
	vorgenommen werden, muss dieses Ansinnen eine Woche vorher
	gegenüber den aktiven Mitgliedern angekündigt werden.
	\item
	Entsteht ein konkreter Änderungsvorschlag auf der Sitzung oder in
	einer hierfür gebildeten Arbeitsgruppe, muss den aktiven
	Mitglieder noch eine Woche Zeit gegeben werden, zu dem konkreten
	Vorschlag Widerspruch zu äußern.
	Sollte Widerspruch geäußert worden sein, so wird auf der
	nächsten Sitzung oder mittels Umlaufverfahren final über den
	Entwurf abgestimmt. Andernfalls wird die Änderung angenommen.
	\item
	Kleine Satzungsänderungen können zusätzlich
	auch über eine schriftliche Abstimmung aller
	aktiven Mitglieder (bspw. per Umlaufverfahren)
	vorgenommen werden. Sollte es Unstimmigkeiten über die
	Interpretation "klein" geben, so muss die Änderung wie jede
	andere behandelt werden.
\end{enumerate}



\section*{§14. Die Mitgliederversammlung}

\begin{enumerate}

	\item
	Die Mitgliederversammlung ist für folgende Angelegenheiten
	zuständig:
	\begin{enumerate}[label=(\roman*)]
		\item
		Entgegennahme des Jahresberichts des Vorstands
		\item
		Entlassung des Vorstandes, Wahl der nächsten Vorstandsmitglieder
		und Abberufung von amtierenden Vorstandsmitgliedern bei außergewöhnlichen
		Umständen
	\end{enumerate}

	\item
	\label{10.1_Stimmrecht}
	Die Mitgliedersammlung ist eine spezielle FSI-Sitzung. \\
		Sitzungsleitung, Stimmrecht, Protokollführung, Art der Abstimmung, Beschlussfähigkeit,
		Fassen von Beschlüssen und Protokollierung von Beschlüssen sind wie in
		\hyperref[12]{§12} geregelt. Sollte keine Beschlussfähigkeit bestehen, muss eine weitere
		Mitgliederversammlung einberufen werden.
\end{enumerate}

\section*{§15. Einberufung der Mitgliederversammlung}

\begin{enumerate}
	\item
	Die Mitgliederversammlung soll einmal jährlich stattfinden.
	Optimalerweise im Zeitraum März – Mai.
	\item
	Die Mitgliederversammlung wird vom Vorstand mit einer Frist von
	4 Wochen schriftlich, fernschriftlich oder in Textform unter
	Verwendung elektronischer Kommunikationsmittel unter Angabe der
	Tagesordnung einberufen. Es ist ausreichend, eine Terminumfrage
	für einen entsprechenden möglichen Zeitraum mit dieser
	4-Wochen-Frist anzugeben. Der endgültige Termin muss 2
	Wochen im Voraus bekannt gegeben werden.
\end{enumerate}


\section*{§16. Außerordentliche Mitgliederversammlung}

\begin{enumerate}
	\item
	Der Vorstand kann jeder Zeit eine außerordentliche
	Mitgliederversammlung einberufen.
	\item
	Eine außerordentliche Mitgliederversammlung ist ferner auf
	schriftliches Verlangen von mindestens 20\% aller
	FSI-Mitglieder binnen 4 Wochen durch den Vorstand
	einzuberufen.
\end{enumerate}


\section*{§17. Gültigkeit dieser Satzung}

\begin{enumerate}
	\item
		Diese Satzung wurde in der Versammlung vom 26.02.2025
		beschlossen.
	\item
		Diese Satzung tritt am \satzungsdatum \ in Kraft.
	\item
		Alle bisherigen Satzungen treten damit außer Kraft.
\end{enumerate}

\end{document}
