\documentclass[a4paper,12pt]{article}
\usepackage[utf8]{inputenc}
\usepackage[ngerman]{babel}
\usepackage{amsmath}
\usepackage{amssymb}
\usepackage{enumitem}
\usepackage{graphicx}

\usepackage{xcolor}

\newcommand{\red}[1]{\textcolor{red}{#1}}
\newcommand{\gray}[1]{\textcolor{gray}{#1}}

\usepackage[hidelinks]{hyperref}


\title{Satzung der FSI Lehramt Südgelände}
\date{11.02.2025}

\begin{document}

\maketitle

\section*{§1. Name}

\begin{enumerate}
    \item Der Name der Fachschaft wurde auf \red{FSI Lehramt Südgelände} festgelegt.
    \item Es wurde sich auf die Abkürzung \red{FSI LA-SÜD} geeinigt.
\end{enumerate}

\section*{§2. Logo}

\begin{enumerate}
	\item
		Es wird folgendes \red{Logo} (und dessen Abwandlungen) verwendet:
		\begin{figure}[h]
			\includegraphics[width=0.5\textwidth]{img/logo.png}
		\end{figure}
		
		\red{TODO}
\end{enumerate}

\section*{§3. Zweck der FSI}

\begin{enumerate}
	\item
		\label{3.1}
		Die FSI soll eine Anlaufstelle für die Studierenden der Fächer \red{Lehramt Biologie, Lehramt Chemie, Lehramt Informatik, Lehramt Mathematik und Lehramt Physik}
		darstellen und diese während ihrer Studienzeit
		unterstützen. Dazu gehört vor allem die Einzelberatung (z.B.
		Eignung und Schwierigkeit einzelner Module) und die allgemeine
		Betreuung von Studierenden (also auch Spaß).
	\item
		Die FSI soll die Studierenden \red{der in \hyperref[3.1]{§3.1} genannten Fächer}
		repräsentieren, auf akute und langfristige Mängel im
		Studienverlauf aufmerksam machen und somit \red{die betreffenden Studienfächer} aus der Sicht von Studierenden vertreten.
	\item
		Die FSI soll für die Studierenden \red{der in \hyperref[3.1]{§3.1} genannten Studienfächer}
		ein Angebot an Aktivitäten, Treffen und Gemeinschaft
		ermöglichen.
\end{enumerate}

\section*{§4. Rechtsgrundlage}

\begin{enumerate}
	\item
		Die FSI ist kein eingetragener Verein und ist daher weder eine \red{
		öffentliche Persönlichkeit ($\rightarrow$ juristische Person ?)} noch rechtlich verankert.
	\item
		Diese Satzung wurde als Konsens erstellt und hält die Ziele,
		Absichten und Abläufe in der FSI fest. Sie ist jedoch nicht
		rechtlich bindend.
    	\item
		Die FSI ist offen für alle Interessierte, unabhängig von deren
		Staatsangehörigkeit, Geschlecht, Herkunft, Religion,
		Weltanschauung, Parteizugehörigkeit und gesellschaftlichen
		Stellung.
\end{enumerate}

\section*{§5. Mitgliederschaft}

\begin{enumerate}
	\item
		Es sind grundsätzlich alle Studierende, ehemalige Studierende
		und Interessierte zu den Veranstaltungen und Aktionen
		eingeladen, falls nicht explizit anders im Voraus gemeldet
		wurde.
    	\item
		Alle Studierenden des Fachs Lehramt Informatik, welche innerhalb
		eines Jahres mindestens drei oder mehr Veranstaltungen
		teilnehmen, können die Mitgliedschaft erhalten. Alternativ kann
		die Mitgliedschaft auch durch einen einstimmigen
		Mitgliederbeschluss erhalten werden.
	\item
		Eine aktive Mitgliedschaft ist genau dann vorhanden, wenn das
		Mitglied unter den Sitzungen der letzten 6 Monate bei mindestens
		50\% anwesend oder vorab entschuldigt war.
	\item
		Mitgliedschaften können durch das Mitglied selbst niedergelegt
		werden.
	\item 	Mitgliedschaften können durch den Vorstand aufgehoben werden.
\end{enumerate}

\section*{§6. Organe der FSI}

\begin{enumerate}
	\item
		Der Vorstand
	\item
		Die FSI Sitzung
	\item
		Die Mitgliederversammlung
\end{enumerate}

\section*{§7. Der Vorstand}

\begin{enumerate}
    	\item
		Der Vorstand besteht aus mindestens einer/einem Vorsitzenden,
		einer/einem stellvertretenden Vorsitzenden. \gray{Auf Beschluss der
		Mitgliederversammlung ist die Ergänzung bis zu 2 weiteren
		Beisitzenden möglich.}
		\red{Wollen wir die Beteiligung jeder Fachrichtung irgendwo aufnehmen oder soll der Vorstand unabhängig sein? Soll es zusätzlich noch eine Studiengangsvertretung geben?}
    	\item
		Der Vorstand hat vor allem folgende Aufgaben
    		\begin{enumerate}[label=(\roman*)]
			\item
				Vorbereiten und Einberufen der
				Mitgliederversammlung sowie das Aufstellen von
				Tagesordnungspunkten \red{der Mitgliederversammlungen und Sitzungen}
        		\item
				Abschluss und Kündigung von Verträgen
        		\item
				Mitgliederverwaltung
    		\end{enumerate}
\end{enumerate}

\section*{§8. Die Bestellung des Vorstandes}

\begin{enumerate}
	\item
		Die Vorsitzenden können nur aktive Studierende (insbesondere
		kein Urlaubssemester) \red{der in \hyperref[3.1]{§3.1} genannten Studienfächer} sein. 
	\item
		\red{Jedes Vorstandsmitglied muss als aktives Mitglied der FSI gelten. \\
			$\rightarrow$ Formulierung unklar, kann man nur Vorstandsmitglied sein, wenn man aktives Mitglied ist oder wird man bei der Wahl zum Vorstand automatisch aktiv - unabhängig von der Anwesenheit? Wie wollen wir es?}
	\item
		Alle Mitglieder des Vorstandes werden während einer
		Mitgliederversammlung \red{für} maximal 1 Jahr gewählt. Jene bleiben
		auch nach einer regulären Amtszeit bis zur nächsten gültigen
		Wahl im Amt.
	\item
		Wiederwahlen sind zulässig.
\end{enumerate}

\section*{§9. Beschlussfassung des Vorstandes}

\begin{enumerate}
    	\item
		Der Vorstand fasst seine Beschlüsse im Allgemeinen in
		Vorstandssitzungen, die schriftlich, fernmündlich oder in
		Textform (bspw. E-Mail) einberufen werden. Jedes
		Vorstandsmitglied ist einberufungsberechtigt.
    		\begin{enumerate}[label=(\roman*)]
			\item
				Eine Einberufungsfrist von einer Woche soll
				eingehalten werden. In dringenden Fällen ist
				eine Einberufung mit kürzerer Frist zulässig.
			\item
				Der Vorstand ist beschlussfähig, wenn
				mindestens zwei Vorstandsmitglieder anwesend
				sind oder schriftlich zustimmen.
			\item
				Bei der Beschlussfassung entscheidet die
				Mehrheit der abgegebenen gültigen Stimmen. Bei
				Stimmengleichheit gilt der Antrag als abgelehnt.
    		\end{enumerate}
	\item
		Vorstandsbeschlüsse können auch schriftlich, fernmündlich oder
		in Textform (bspw. E-Mail) gefasst werden, wenn die absolute
		Mehrheit der Vorstandsmitglieder ihre Zustimmung zu der zu
		beschließenden Regelung erklären.
\end{enumerate}

\section*{§10. Mitgliederversammlung}

\begin{enumerate}
	\item
		\label{10.1_Stimmrecht}
		Stimmrecht: \textcolor{magenta}{Umstrukturierungsvorschlag: Da das unabhängig von der Mitgliederversammlung ist und auch für die FSI-Sitzung gilt, würde ich das Stimmrecht hier rausnehmen, es zur FSI-Sitzung packen und hier schreiben, dass die Mitgliederversammlung eine spezielle FSI-Sitzung ist und daher das Stimmrecht wie in ... geregelt ist.}
		\begin{enumerate}[label=(\roman*)]
        		\item
				Jedes aktive Mitglied hat eine Stimme.
        		\item
				Zur Ausübung des Stimmrechts kann auch ein
				anderes Mitglied schriftlich oder in Textform
				(bspw. E-Mail) bevollmächtigt werden.
        		\item
				Die Bevollmächtigung ist für jede
				Mitgliederversammlung gesondert zu erteilen.
        		\item
				Ein Mitglied darf nicht mehr als drei
				fremde Stimmen vertreten.
    		\end{enumerate}
    	\item
		Die Mitgliederversammlung ist für folgende Angelegenheiten
		zuständig:
		\begin{enumerate}[label=(\roman*)]
			\item
				Entgegennahme des Jahresberichts des Vorstands
			\item
				Entlassung des Vorstandes, die Wahl der \red{nächsten Vorstandsmitglieder}
				und deren Abberufung bei außergewöhnlichen
				Umständen \red{$\rightarrow$ was wollen wir mit dem letzten Teil sagen?}
			\item
				Beschlussfassungen über die Änderungen der
				Satzung. Satzungsänderungen können zusätzlich
				auch über eine schriftliche Abstimmung aller
				aktiven Mitglieder (bspw. per Umlaufverfahren)
				vorgenommen werden.
		\end{enumerate}
\end{enumerate}

\section*{§11. Die Einberufung der Mitgliederversammlung}

\begin{enumerate}
    	\item
		Die Mitgliederversammlung soll einmal jährlich stattfinden.
		Optimalerweise im Zeitraum August – Oktober.
    	\item
		Die Mitgliederversammlung wird vom Vorstand mit einer Frist von
		4 Wochen schriftlich, fernschriftlich oder in Textform unter
		Verwendung elektronischer Kommunikationsmittel unter Angabe der
		Tagesordnung einberufen. Es ist ausreichend, eine Terminumfrage
		für einen entsprechenden möglichen Zeitraum mit dieser
		4-Wochen-Frist anzugeben. Der endgültige Termin muss 2
		Wochen im Voraus bekannt gegeben werden.
    	\item
		Die Tagesordnung setzt der Vorstand fest. Allerdings können
		aktive Mitglieder Beiträge/Ergänzungen zur Tagesordnung der
		Mitgliederversammlung bis zu einer Woche vor der
		Mitgliederversammlung einreichen.
\end{enumerate}

\section*{§12. Die Beschlussfassung der Mitgliederversammlung}

\textcolor{magenta}{12.1 bis 12.3 sind in der FSI-Sitzung genauso (wird also direkt referenziert), hier aber mit der Wortwahl der Mitgliederversammlung geschrieben, also suboptimal. Mit der obigen pinken Änderung, dass die Mitgliederversammlung eine spezielle FSI-Sitzung ist, könnte man hier 12.1 - 12.3 zur FSI Sitzung packen und hier dafür schreiben, dass eben diese drei Abschnitte (Sitzungsleitung, Protokollführung und  Art der Abstimmung) hier (für die Mitgliederversammlung) genauso gelten. \\
Für 12.5 (Fassen von Beschlüssen) und 12.6 (Protokoll $\rightarrow$ hier plus Ergänzung zu Satzungsänderungen) gilt das Gleiche.}
	
\begin{enumerate}
		\label{12.1}
    	\item
		Die Mitgliederversammlung wird von \gray{der/dem Vorsitzenden oder bei
		deren/dessen Verhinderung von dem/der stellvertretenden
		Vorsitzenden} geleitet. Ist kein Vorstandsmitglied anwesend,
		bestimmt die Mitgliederversammlung die/den Leitenden.
		\red{$\rightarrow$ je nach Vorstandsstruktur muss hier die graue Formulierung angepasst werden}
    	\item
		Die/der Protokollführende wird \red{von der Versammlungsleitung} bestimmt.
		Diese/dieser muss kein aktives Mitglied sein.
    	\item
		Die Art der Abstimmung wird durch \red{die Versammlungsleitung} bestimmt. \red{Die Art der Abstimmung}
		ist, falls nicht anders bestimmt, offen. Auf Antrag eines
		Mitgliedes ist geheim abzustimmen.
    	\item
		Die Mitgliederversammlung ist beschlussfähig, wenn 4 \red{($\rightarrow$ falls die FSI deutlich größer wird, lässt sich über diese kleine Zahl vielleicht streiten, evtl. können wir es aber erstmal so lassen)} oder 50\%
		aller stimmberechtigten \red{Mitglieder} anwesend oder
		ordnungsgemäß vertreten sind. Besteht keine Beschlussfähigkeit,
		so muss eine weitere Mitgliederversammlung einberufen werden.
    	\item
		Fassen von Beschlüssen:
    		\begin{enumerate}[label=(\roman*)]
			\item
				Im Allgemeinen wird ein Beschluss mit einfacher
				Mehrheit der abgegebenen gültigen Stimmen
				gefasst.
			\item
				Enthaltungen gelten nicht als abgegebene Stimme.
			\item
				Beschlüsse über Satzungsänderungen benötigen
				eine Zweidrittelmehrheit der abgegebenen
				Stimmen.
			\item
				Die Abwahl eines Vorstandsmitgliedes benötigt
				ebenfalls eine Zweidrittelmehrheit der
				\red{abgegebenen Stimmen}.
		\end{enumerate}
    	\item
		Über die Beschlüsse ist ein Protokoll anzufertigen, welches für
		Mitglieder einsehbar abgelegt oder per Mail zugesandt wird.
		Es soll enthalten:
    		\begin{enumerate}[label=(\roman*)]
        		\item
				Ort und Zeit der Versammlung
        		\item
				Anwesende Personen sowie Zahl der erschienenen
				\red{Mitglieder}
        		\item
				Die Abstimmungsergebnisse und Beschlüsse
        		\item
				Bei Satzungsänderungen soll der genaue Wortlaut
				\red{der Änderungen} angegeben werden
    		\end{enumerate}
\end{enumerate}

\section*{§13. Außerordentliche Mitgliederversammlung}

\begin{enumerate}
	\item
		Der Vorstand kann jeder Zeit eine außerordentliche
		Mitgliederversammlung einberufen.
    	\item
		Eine außerordentliche Mitgliederversammlung ist ferner auf
		schriftliches Verlangen von mindestens 20\% aller
		\red{FSI-Mitglieder} binnen 4 Wochen durch den Vorstand
		einzuberufen.
\end{enumerate}

\section*{§14. Die FSI Sitzung}

\begin{enumerate}
    	\item
		Das Stimmrecht ist wie in \hyperref[10.1_Stimmrecht]{§10.1} gegeben.
    	\item
		Die FSI Sitzung ist für das „Tagesgeschäft“ und die
		dazugehörigen Entscheidungen zuständig.
\end{enumerate}

\section*{§15. Die Einberufung der FSI Sitzung}

\begin{enumerate}
    	\item
		Die FSI Sitzung soll alle zwei bis drei Wochen stattfinden. \red{$\rightarrow$ über diese Frist muss man sicher mit den anderen noch reden, erstmal kann man es aber so lassen, es ist schließlich auch nur ein \glqq soll\grqq{} und kein \glqq muss\grqq{}}
    	\item
		Die FSI Sitzung wird vom Vorstand schriftlich, fernschriftlich
		oder in Textform unter Verwendung elektronischer
		Kommunikationsmittel einberufen. Er wird dazu angehalten, dies
		mindestens eine Woche im Voraus zu tun.
    	\item
		Die Tagesordnung setzt der Vorstand fest. Allerdings können
		aktive Mitglieder, ebenso wie externe Gäste und Interessierte,
		Beiträge/Ergänzungen zur Tagesordnung über den Vorstand
		einbringen. Diese müssen innerhalb der nächsten 2 Sitzungen
		berücksichtigt werden.
\end{enumerate}

\section*{§16. Die Beschlussfähigkeit der FSI Sitzung}

\begin{enumerate}
    	\item
		\hyperref[12.1]{§12.1} bis §12.3 gelten auch hier \red{(genau analog für die FSI Sitzung)}.
    	\item
		Die FSI Sitzung ist beschlussfähig, wenn 4 oder 50\% aller
		stimmberechtigten Mitglieder anwesend oder ordnungsgemäß
		vertreten sind. Besteht keine Beschlussfähigkeit, so können
		keine Beschlüsse gefasst werden.
    	\item
		Das Fassen von Beschlüssen wird aus §12.5 übernommen.
    	\item
		Das Protokollieren gilt wie in §12.6.
\end{enumerate}

\section*{§17. Gültigkeit dieser Satzung}

\begin{enumerate}
	\item
		Diese Satzung wurde in der Versammlung vom \red{05.10.2022 $\rightarrow$ TODO}
		beschlossen.
	\item
		Alle bisherigen Satzungen treten damit außer Kraft.
\end{enumerate}

\end{document}
