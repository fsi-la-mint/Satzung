\documentclass[a4paper,12pt]{article}
\usepackage[utf8]{inputenc}
\usepackage[ngerman]{babel}
\usepackage{amsmath}
\usepackage{amssymb}
\usepackage{enumitem}
\usepackage{graphicx}

\title{Satzung der FSI Lehramt Informatik}
\date{05.10.2022}

\begin{document}

\maketitle

\section*{§1. Name}

\begin{enumerate}
    \item Der Name der Fachschaft wurde durch Abstimmung auf FSI Lehramt Informatik festgelegt.
    \item Es wurde sich auf die Abkürzung FSI LA-INF geeinigt.
\end{enumerate}

\section*{§2. Logo}

\begin{enumerate}
	\item
		Es wird folgendes Logo (und dessen Abwandlungen) verwendet:
		\begin{figure}[h]
			\includegraphics[width=0.5\textwidth]{img/logo.png}
		\end{figure}
\end{enumerate}

\section*{§3. Zweck der FSI}

\begin{enumerate}
	\item
		Die FSI soll eine Anlaufstelle für die Studierenden des Fachs
		Lehramt Informatik darstellen und sie während ihrer Studienzeit
		unterstützen. Dazu gehört vor allem die Einzelberatung (z.B.
		Eignung und Schwierigkeit einzelner Module) und die allgemeine
		Betreuung von Studierenden (also auch Spaß).
	\item
		Die FSI soll die Studierenden des Fachs Lehramt Informatik
		repräsentieren, auf akute und langfristige Mängel im
		Studienverlauf aufmerksam machen und somit das Studienfach
		Lehramt Informatik aus der Sicht eines Studierenden vertreten.
	\item
		Die FSI soll für die Studierenden des Fachs Lehramt Informatik
		ein Angebot an Aktivitäten, Treffen und Gemeinschaft
		ermöglichen.
\end{enumerate}

\section*{§4. Rechtsgrundlage}

\begin{enumerate}
	\item
		Die FSI ist kein eingetragener Verein und ist daher weder eine
		öffentliche Persönlichkeit noch rechtlich verankert.
	\item
		Diese Satzung wurde als Konsens erstellt und hält die Ziele,
		Absichten und Abläufe in der FSI fest. Sie ist aber nicht
		rechtlich bindend.
    	\item
		Die FSI ist offen für alle Interessierte, unabhängig von deren
		Staatsangehörigkeit, Geschlecht, Herkunft, Religion,
		Weltanschauung, Parteizugehörigkeit und gesellschaftlichen
		Stellung.
\end{enumerate}

\section*{§5. Mitgliederschaft}

\begin{enumerate}
	\item
		Es sind grundsätzlich alle Studierende, ehemalige Studierende
		und Interessierte zu den Veranstaltungen und Aktionen
		eingeladen, falls nicht explizit anders im Voraus gemeldet
		wurde.
    	\item
		Alle Studierenden des Fachs Lehramt Informatik, welche innerhalb
		eines Jahres mindestens drei oder mehr Veranstaltungen
		teilnehmen, können die Mitgliedschaft erhalten. Alternativ kann
		die Mitgliedschaft auch durch einen einstimmigen
		Mitgliederbeschluss erhalten werden.
	\item
		Eine aktive Mitgliedschaft ist genau dann vorhanden, wenn das
		Mitglied unter den Sitzungen der letzten 6 Monate bei mindestens
		50\% anwesend oder vorab entschuldigt war.
	\item
		Mitgliedschaften können durch das Mitglied selbst niedergelegt
		werden.
	\item 	Mitgliedschaften können durch den Vorstand aufgehoben werden.
\end{enumerate}

\section*{§6. Organe der FSI}

\begin{enumerate}
	\item
		Der Vorstand
	\item
		Die FSI Sitzung
	\item
		Die Mitgliederversammlung
\end{enumerate}

\section*{§7. Der Vorstand}

\begin{enumerate}
    	\item
		Der Vorstand besteht aus mindestens einer/einem Vorsitzenden,
		einer/einem stellvertretenden Vorsitzenden. Auf Beschluss der
		Mitgliederversammlung ist die Ergänzung bis zu 2 weiteren
		Beisitzenden möglich.
    	\item
		Der Vorstand hat vor allem folgende Aufgaben
    		\begin{enumerate}[label=(\roman*)]
			\item
				Vorbereiten und Einberufen der
				Mitgliederversammlung sowie das Aufstellen von
				Tagesordnungspunkten
        		\item
				Abschluss und Kündigung von Verträgen
        		\item
				Mitgliederverwaltung
    		\end{enumerate}
\end{enumerate}

\section*{§8. Die Bestellung des Vorstandes}

\begin{enumerate}
	\item
		Die Vorsitzenden können nur aktive Studierende (insbesondere
		kein Urlaubssemester) des Fachs Lehramt Informatik sein.
	\item
		Jegliche Vorstandsmitgliederinnen/Vorstandsmitglieder müssen
		als aktives Mitglied der FSI gelten.
	\item
		Alle Mitglieder des Vorstandes werden während einer
		Mitgliederversammlung auf maximal 1 Jahr gewählt. Jene bleiben
		auch nach einer regulären Amtszeit bis zur nächsten gültigen
		Wahl im Amt.
	\item
		Wiederwahlen sind zulässig.
\end{enumerate}

\section*{§9. Beschlussfassung des Vorstandes}

\begin{enumerate}
    	\item
		Der Vorstand fasst seine Beschlüsse im Allgemeinen in
		Vorstandssitzungen, die schriftlich, fernmündlich oder in
		Textform (bspw. E-Mail) einberufen werden. Jedes
		Vorstandsmitglied ist einberufungsberechtigt.
    		\begin{enumerate}[label=(\roman*)]
			\item
				Eine Einberufungsfrist von einer Woche soll
				eingehalten werden. In dringenden Fällen ist
				eine Einberufung mit kürzerer Frist zulässig.
			\item
				Der Vorstand ist beschlussfähig, wenn
				mindestens zwei Vorstandsmitglieder anwesend
				sind oder schriftlich zustimmen.
			\item
				Bei der Beschlussfassung entscheidet die
				Mehrheit der abgegebenen gültigen Stimmen. Bei
				Stimmengleichheit gilt der Antrag als abgelehnt.
    		\end{enumerate}
	\item
		Vorstandsbeschlüsse können auch schriftlich, fernmündlich oder
		in Textform (bspw. E-Mail) gefasst werden, wenn die absolute
		Mehrheit der Vorstandsmitglieder ihre Zustimmung zu der zu
		beschließenden Regelung erklären.
\end{enumerate}

\section*{§10. Mitgliederversammlung}

\begin{enumerate}
	\item
		Stimmrecht:
		\begin{enumerate}[label=(\roman*)]
        		\item
				Jedes aktive Mitglied hat eine Stimme.
        		\item
				Zur Ausübung des Stimmrechts kann auch ein
				anderes Mitglied schriftlich oder in Textform
				(bspw. E-Mail) bevollmächtigt werden.
        		\item
				Die Bevollmächtigung ist für jede
				Mitgliederversammlung gesondert zu erteilen.
        		\item
				Ein Mitglied darf jedoch nicht mehr als drei
				fremde Stimmen vertreten.
    		\end{enumerate}
    	\item
		Die Mitgliederversammlung ist für folgende Angelegenheiten
		zuständig:
		\begin{enumerate}[label=(\roman*)]
			\item
				Entgegennahme des Jahresberichts des Vorstands
			\item
				Entlassung des Vorstandes, die Wahl der
				einzelnen
				Vorstandsmitgliederinnen/Vorstandsmitglieder
				und deren Abberufung bei außergewöhnlichen
				Umständen
			\item
				Beschlussfassungen über die Änderungen der
				Satzung. Satzungsänderungen können zusätzlich
				auch über eine schriftliche Abstimmung aller
				aktiven Mitglieder (bspw. per Umlaufverfahren)
				vorgenommen werden.
		\end{enumerate}
\end{enumerate}

\section*{§11. Die Einberufung der Mitgliederversammlung}

\begin{enumerate}
    	\item
		Die Mitgliederversammlung soll einmal jährlich stattfinden.
		Optimalerweise im Zeitraum August – Oktober.
    	\item
		Die Mitgliederversammlung wird vom Vorstand mit einer Frist von
		4 Wochen schriftlich, fernschriftlich oder in Textform unter
		Verwendung elektronischer Kommunikationsmittel unter Angabe der
		Tagesordnung einberufen. Es ist ausreichend, eine Terminumfrage
		für einen entsprechenden möglichen Zeitraum mit dieser
		4-Wochen-Frist anzugeben. Der endgültige Termin muss dann 2
		Wochen im Voraus bekannt gegeben werden.
    	\item
		Die Tagesordnung setzt der Vorstand fest. Allerdings können
		aktive Mitglieder Beiträge/Ergänzungen zur Tagesordnung der
		Mitgliederversammlung bis zu einer Woche vor der
		Mitgliederversammlung einreichen.
\end{enumerate}

\section*{§12. Die Beschlussfassung der Mitgliederversammlung}

\begin{enumerate}
    	\item
		Die Mitgliederversammlung wird von der/dem Vorsitzenden oder bei
		deren/dessen Verhinderung von dem/der stellvertretenden
		Vorsitzenden geleitet. Ist kein Vorstandsmitglied anwesend,
		bestimmt die Mitgliederversammlung die/den Leitenden.
    	\item
		Die/der Protokollführende wird vom Versammlungsleiter bestimmt.
		Diese/dieser muss kein aktives Mitglied sein.
    	\item
		Die Art der Abstimmung wird durch den Leiter bestimmt. Diese
		ist, falls nicht anders bestimmt, offen. Auf Antrag eines
		Mitgliedes ist geheim abzustimmen.
    	\item
		Die Mitgliederversammlung ist beschlussfähig, wenn 4 oder 50\%
		aller stimmberechtigten Mitgliederinnen/Mitglieder anwesend oder
		ordnungsgemäß vertreten sind. Besteht keine Beschlussfähigkeit,
		so muss eine weitere Mitgliederversammlung einberufen werden.
    	\item
		Fassen von Beschlüssen:
    		\begin{enumerate}[label=(\roman*)]
			\item
				Im Allgemeinen wird ein Beschluss mit einfacher
				Mehrheit der abgegebenen gültigen Stimmen
				gefasst.
			\item
				Enthaltungen gelten nicht als abgegebene Stimme.
			\item
				Beschlüsse über Satzungsänderungen benötigen
				eine Zweidrittelmehrheit der abgegebenen
				Stimmen.
			\item
				Die Abwahl eines Vorstandsmitgliedes benötigt
				ebenfalls eine Zweidrittelmehrheit der
				Anwesenden.
		\end{enumerate}
    	\item
		Über die Beschlüsse ist ein Protokoll anzufertigen, welches für
		Mitglieder einsehbar abgelegt wird oder per Mail zugesandt wird.
		Es soll enthalten:
    		\begin{enumerate}[label=(\roman*)]
        		\item
				Ort und Zeit der Versammlung
        		\item
				Anwesende Personen sowie Zahl der erschienenen
				Mitgliederinnen/Mitglieder
        		\item
				Die Abstimmungsergebnisse und Beschlüsse
        		\item
				Bei Satzungsänderungen soll der genaue Wortlaut
				angegeben werden
    		\end{enumerate}
\end{enumerate}

\section*{§13. Außerordentliche Mitgliederversammlung}

\begin{enumerate}
	\item
		Der Vorstand kann jeder Zeit eine außerordentliche
		Mitgliederversammlung einberufen.
    	\item
		Eine außerordentliche Mitgliederversammlung ist ferner auf
		schriftliches Verlangen von mindestens 20\% aller
		Vereinsmitglieder binnen 4 Wochen durch den Vorstand
		einzuberufen.
\end{enumerate}

\section*{§14. Die FSI Sitzung}

\begin{enumerate}
    	\item
		Das Stimmrecht ist wie in §10.1 gegeben.
    	\item
		Die FSI Sitzung ist für das „Tagesgeschäft“ und die
		dazugehörigen Entscheidungen zuständig.
\end{enumerate}

\section*{§15. Die Einberufung der FSI Sitzung}

\begin{enumerate}
    	\item
		Die FSI Sitzung soll alle zwei bis drei Wochen stattfinden.
    	\item
		Die FSI Sitzung wird vom Vorstand schriftlich, fernschriftlich
		oder in Textform unter Verwendung elektronischer
		Kommunikationsmittel einberufen. Er wird dazu angehalten, dies
		mindestens eine Woche im Voraus zu tun.
    	\item
		Die Tagesordnung setzt der Vorstand fest. Allerdings können
		aktive Mitglieder, ebenso wie externe Gäste und Interessierte,
		Beiträge/Ergänzungen zur Tagesordnung über den Vorstand
		einbringen. Diese müssen innerhalb der nächsten 2 Sitzungen
		berücksichtigt werden.
\end{enumerate}

\section*{§16. Die Beschlussfähigkeit der FSI Sitzung}

\begin{enumerate}
    	\item
		§12.1 bis §12.3 gelten auch hier (für eben genau die FSI
		Sitzung).
    	\item
		Die FSI Sitzung ist beschlussfähig, wenn 4 oder 50\% aller
		stimmberechtigten Mitglieder anwesend oder ordnungsgemäß
		vertreten sind. Besteht keine Beschlussfähigkeit, so können
		keine Beschlüsse gefasst werden.
    	\item
		Das Fassen von Beschlüssen wird aus §12.5 übernommen.
    	\item
		Das Protokollieren gilt wie in §12.6.
\end{enumerate}

\section*{§17. Gültigkeit dieser Satzung}

\begin{enumerate}
	\item
		Diese Satzung wurde in der Versammlung vom 05.10.2022
		beschlossen.
	\item
		Alle bisherigen Satzungen treten damit außer Kraft.
\end{enumerate}

\end{document}
