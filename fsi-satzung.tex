\documentclass[a4paper,12pt]{article}
\usepackage[utf8]{inputenc}
\usepackage[ngerman]{babel}
\usepackage{amsmath}
\usepackage{amssymb}
\usepackage{enumitem}
\usepackage{graphicx}

\usepackage{xcolor}

\newcommand{\red}[1]{\textcolor{red}{#1}}
\newcommand{\gray}[1]{\textcolor{gray}{#1}}

\usepackage[hidelinks]{hyperref}


\title{Satzung der FSI Lehramt Südgelände}
\date{11.02.2025}

\begin{document}

\maketitle

\section*{§1. Name}

\begin{enumerate}
    \item Der Name der Fachschaft wurde auf \red{FSI Lehramt Südgelände} festgelegt.
    \item Es wurde sich auf die Abkürzung \red{FSI LA-SÜD} geeinigt.
\end{enumerate}

\section*{§2. Logo}

\begin{enumerate}
	\item
		Es wird folgendes \red{Logo} (und dessen Abwandlungen) verwendet:
		\begin{figure}[h]
			\includegraphics[width=0.5\textwidth]{img/logo.png}
		\end{figure}

		\red{TODO}
\end{enumerate}

\section*{§3. Zweck der FSI}

\begin{enumerate}
	\item
		\label{3.1}
		Die FSI soll eine Anlaufstelle für die Studierenden der Fächer \red{Lehramt Biologie, Lehramt Chemie, Lehramt Informatik, Lehramt Mathematik und Lehramt Physik}
		darstellen und diese während ihrer Studienzeit
		unterstützen. Dazu gehört vor allem die Einzelberatung (z.B.
		Eignung und Schwierigkeit einzelner Module) und die allgemeine
		Betreuung von Studierenden (also auch Spaß).
	\item
		Die FSI soll die Studierenden \red{der in \hyperref[3.1]{§3.1} genannten Fächer}
		repräsentieren, auf akute und langfristige Mängel im
		Studienverlauf aufmerksam machen und somit \red{die betreffenden Studienfächer} aus der Sicht von Studierenden vertreten.
	\item
		Die FSI soll für die Studierenden \red{der in \hyperref[3.1]{§3.1} genannten Studienfächer}
		ein Angebot an Aktivitäten, Treffen und Gemeinschaft
		ermöglichen.
		\item 
		\red{Es können nur diejenigen der in \hyperref[3.1]{§3.1} genannten Fächer durch die FSI vertreten werden, von denen auch Studierende in der FSI-Mitglied sind.}

\end{enumerate}

\section*{§4. Rechtsgrundlage}

\begin{enumerate}
	\item
		Die FSI ist kein eingetragener Verein und ist daher weder eine \red{
		juristische Person} noch rechtlich verankert.
	\item
		Diese Satzung wurde als Konsens erstellt und hält die Ziele,
		Absichten und Abläufe in der FSI fest. Sie ist jedoch nicht
		rechtlich bindend.
    	\item
		Die FSI ist offen für alle Interessierte, unabhängig von deren
		Staatsangehörigkeit, Geschlecht, Herkunft, Religion,
		Weltanschauung, Parteizugehörigkeit und gesellschaftlichen
		Stellung.
\end{enumerate}

\section*{§5. Mitgliedschaft}

\begin{enumerate}
	\item
		Es sind grundsätzlich alle Studierende, ehemalige Studierende
		und Interessierte zu den Veranstaltungen und Aktionen
		eingeladen, falls nicht explizit anders im Voraus gemeldet
		wurde.
    	\item
		Alle Studierenden \red{der in \hyperref[3.1]{§3.1} genannten Studienfächer}, welche innerhalb
		eines Jahres mindestens drei oder mehr Veranstaltungen
		teilnehmen, können die Mitgliedschaft erhalten. Alternativ kann
		die Mitgliedschaft auch durch einen einstimmigen
		Mitgliederbeschluss erhalten werden.
	\item
		Eine aktive Mitgliedschaft ist genau dann vorhanden, wenn das
		Mitglied unter den Sitzungen der letzten 6 Monate bei mindestens
		50\% anwesend oder vorab entschuldigt war.
	\item
		Mitgliedschaften können durch das Mitglied selbst niedergelegt
		werden.
	\item 	Mitgliedschaften können durch den Vorstand aufgehoben werden.
\end{enumerate}

\section*{§6. Organe der FSI}

\begin{enumerate}
	\item
		Der Vorstand
	\item
		Die FSI-Sitzung
\end{enumerate}

\section*{§7. Der Vorstand}

\begin{enumerate}
	\item 
	Der Vorstand besteht aus mindestens einer/einem Vorsitzenden, einer/einem stellvertretenden Vorsitzenden. \red{Zudem ist jeweils die Studienfachvertretung jeder Fachrichtung, die derzeit von der FSI vertreten wird, automatisch Mitglied im Vorstand.} 
	\item
	Der Vorstand hat vor allem folgende Aufgaben:
		\begin{enumerate}[label=(\roman*)]
		\item
			Vorbereiten und Einberufen, sowie das Aufstellen von
			Tagesordnungspunkten \red{der FSI-Sitzungen}
			\item
			Abschluss und Kündigung von Verträgen
			\item
			Mitgliederverwaltung
		\end{enumerate}
\end{enumerate}

\section*{§8. Die Bestellung des Vorstandes}

\begin{enumerate}
	\item
		Die Vorsitzenden können nur aktive Studierende (insbesondere
		kein Urlaubssemester) \red{der in \hyperref[3.1]{§3.1} genannten Studienfächer} sein.
	\item
		\red{
		Studienfachvertretung:
		\begin{enumerate}[label=(\roman*)]
			\item 
			Jedes Studienfach, welches derzeit von der FSI vertreten wird, besitzt eine Studienfachvertreterin oder einen Studienfachvertreter.
			\item 
			Die Studienfachvertretung wird per relativer Mehrheit aller derzeit das jeweilige Studienfach studierenden FSI-Mitglieder gewählt.
			\item 
			Mehrfachbesetzungen aus Studienfachvertretung eines oder mehrerer Fächer und/oder (stellvertretendem) FSI-Vorsitz sind zulässig.			
		\end{enumerate}}
	\item
		Alle Mitglieder des Vorstandes werden während einer
		FSI-Sitzung, \red{die mindestens zwei Wochen im Voraus über die üblichen Kanäle 
		angekündigt werden muss,} \red{für} maximal 1 Jahr gewählt. Jene bleiben
		auch nach einer regulären Amtszeit bis zur nächsten gültigen
		Wahl im Amt.
	\item
		Wiederwahlen sind zulässig.
\end{enumerate}

\section*{§9. Beschlussfassung des Vorstandes}

\begin{enumerate}
    	\item
		Der Vorstand fasst seine Beschlüsse im Allgemeinen in
		Vorstandssitzungen, die schriftlich, fernmündlich oder in
		Textform (bspw. E-Mail) einberufen werden. Jedes
		Vorstandsmitglied ist einberufungsberechtigt.
    		\begin{enumerate}[label=(\roman*)]
			\item
				Eine Einberufungsfrist von einer Woche soll
				eingehalten werden. In dringenden Fällen ist
				eine Einberufung mit kürzerer Frist zulässig.
			\item
				Der Vorstand ist beschlussfähig, wenn
				mindestens zwei Vorstandsmitglieder anwesend
				sind oder schriftlich zustimmen.
			\item
				Bei der Beschlussfassung entscheidet die
				Mehrheit der abgegebenen gültigen Stimmen. Bei
				Stimmengleichheit gilt der Antrag als abgelehnt.
    		\end{enumerate}
	\item
		Vorstandsbeschlüsse können auch schriftlich, fernmündlich oder
		in Textform (bspw. E-Mail) gefasst werden, wenn die absolute
		Mehrheit der Vorstandsmitglieder ihre Zustimmung zu der zu
		beschließenden Regelung erklären.
\end{enumerate}

\section*{§10. FSI-Sitzungen}

	Die FSI-Sitzung ist für das „Tagesgeschäft“ und die
	dazugehörigen Entscheidungen zuständig.


\section*{§11. Die Einberufung der FSI-Sitzung}

\begin{enumerate}
	\item
		\red{Die FSI-Sitzung soll regelmäßig stattfinden.}
	\item
		\red{Entsprechende Ankündigungen sollen eine Woche vorher auf den
		üblichen Kanälen verbreitet werden.}
    	\item
		Die Tagesordnung setzt der Vorstand fest. Allerdings können
		aktive Mitglieder Beiträge/Ergänzungen zur Tagesordnung der
		FSI-Sitzung bis zu \red{zwei Tage} vor der
		FSI-Sitzung einreichen. \red{Ob spontane Themen auf der
		Sitzung behandelt werden, entscheidet die Sitzungsleitung.}
\end{enumerate}



\section*{§12. Beschlussfähigkeit der FSI-Sitzung}
\begin{enumerate}
	\item
	Eine FSI-Sitzung wird von \red{einer/einem der Vorstandsmitglieder geleitet. Ist kein Vorstandsmitglied anwesend,
	bestimmen die anwesenden FSI-Mitglieder} die/den Leitenden.
	\item
	\label{10.1_Stimmrecht}
	Stimmrecht:
	\begin{enumerate}[label=(\roman*)]
		\item
		\red{Sofern nicht anders geregelt, hat jedes FSI-Mitglied} eine Stimme.
		\item
		Zur Ausübung des Stimmrechts kann auch ein
		anderes Mitglied schriftlich oder in Textform
		(bspw. E-Mail) bevollmächtigt werden.
		\item
		Die Bevollmächtigung ist für jede
		\red{FSI-Sitzung} gesondert zu erteilen.
		\item
		Ein Mitglied darf nicht mehr als drei
		fremde Stimmen vertreten.
	\end{enumerate}
	\item
	Die/der Protokollführende wird \red{von der Sitzungsleitung} bestimmt.
	Diese/dieser muss kein aktives Mitglied sein.
	\item
	Die Art der Abstimmung wird durch \red{die Sitzungsleitung} bestimmt. \red{Die Art der Abstimmung}
	ist, falls nicht anders bestimmt, offen. Auf Antrag eines
	Mitgliedes ist geheim abzustimmen.
	\item
	\red{Die FSI-Sitzung ist beschlussfähig, wenn 50\%
	aller stimmberechtigten \red{Mitglieder} anwesend oder
	ordnungsgemäß vertreten sind. Besteht keine Beschlussfähigkeit,
	so so können keine Beschlüsse gefasst werden.}
	\item
	Fassen von Beschlüssen:
	\begin{enumerate}[label=(\roman*)]
		\item
		Im Allgemeinen wird ein Beschluss mit einfacher
		Mehrheit der abgegebenen gültigen Stimmen
		gefasst.
		\item
		Enthaltungen gelten nicht als abgegebene Stimme.
		\item
		Beschlüsse über Satzungsänderungen benötigen
		eine Zweidrittelmehrheit der abgegebenen
		Stimmen.
		\item
		Die Abwahl eines Vorstandsmitgliedes benötigt
		ebenfalls eine Zweidrittelmehrheit der
		\red{abgegebenen Stimmen}.
	\end{enumerate}
	\item
	Über die Beschlüsse ist ein Protokoll anzufertigen, welches für
	Mitglieder einsehbar abgelegt oder per Mail zugesandt wird.
	Es soll enthalten:
	\begin{enumerate}[label=(\roman*)]
		\item
		Ort und Zeit der Versammlung
		\item
		Anwesende Personen sowie Zahl der erschienenen
		\red{Mitglieder}
		\item
		Die Abstimmungsergebnisse und Beschlüsse
		\item
		Bei Satzungsänderungen soll der genaue Wortlaut
		\red{der Änderungen} angegeben werden
	\end{enumerate}
\end{enumerate}

%TODO Formulierung Vorstandsaufbau -> done
%TODO Restliche Anpassung von Mitgliederversammlung/FSI Sitzung -> FSI-Sitzung -> done
%TODO Vorstandswahl alle Elemente, 2 Wochen Vorankündigung -> done

\section*{§13. Satzungsänderungen}
\begin{enumerate}
	\item
		Soll eine Satzungsänderung während einer FSI-Sitzung
		vorgenommen werden, muss dieses Ansinnen eine Woche vorher
		gegenüber den aktiven Mitgliedern angekündigt werden.
	\item
		Entsteht ein konreter Änderungsvorschlag auf der Sitzung oder in
		einer hierfür gebildeten Arbeitsgruppe, muss den aktiven
		Mitglieder noch eine Woche Zeit gegeben werden, zu dem konkreten
		Vorschlag Widerspruch zu äußern.
		Sollte Widerspruch geäußert worden sein, so wird auf der
		nächsten Sitzung oder mittels Umlaufverfahren final über den
		Entwurf abgestimmt. Andernfalls wird die Änderung angenommen.
	\item
		Kleine Satzungsänderungen können zusätzlich
		auch über eine schriftliche Abstimmung aller
		aktiven Mitglieder (bspw. per Umlaufverfahren)
		vorgenommen werden. Sollte es Unstimmigkeiten über die
		Interpretation "klein" geben, so muss die Änderung wie jede
		andere behandelt werden.
\end{enumerate}



\section*{§17. Gültigkeit dieser Satzung}

\begin{enumerate}
	\item
		Diese Satzung wurde in der Versammlung vom \red{05.10.2022 $\rightarrow$ TODO}
		beschlossen.
	\item
		Alle bisherigen Satzungen treten damit außer Kraft.
\end{enumerate}

\end{document}
